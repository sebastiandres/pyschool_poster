% Options for packages loaded elsewhere
\PassOptionsToPackage{unicode}{hyperref}
\PassOptionsToPackage{hyphens}{url}
\PassOptionsToPackage{dvipsnames,svgnames,x11names}{xcolor}
%
\documentclass[
  letterpaper,
  DIV=11,
  numbers=noendperiod]{scrartcl}

\usepackage{amsmath,amssymb}
\usepackage{iftex}
\ifPDFTeX
  \usepackage[T1]{fontenc}
  \usepackage[utf8]{inputenc}
  \usepackage{textcomp} % provide euro and other symbols
\else % if luatex or xetex
  \usepackage{unicode-math}
  \defaultfontfeatures{Scale=MatchLowercase}
  \defaultfontfeatures[\rmfamily]{Ligatures=TeX,Scale=1}
\fi
\usepackage{lmodern}
\ifPDFTeX\else  
    % xetex/luatex font selection
\fi
% Use upquote if available, for straight quotes in verbatim environments
\IfFileExists{upquote.sty}{\usepackage{upquote}}{}
\IfFileExists{microtype.sty}{% use microtype if available
  \usepackage[]{microtype}
  \UseMicrotypeSet[protrusion]{basicmath} % disable protrusion for tt fonts
}{}
\makeatletter
\@ifundefined{KOMAClassName}{% if non-KOMA class
  \IfFileExists{parskip.sty}{%
    \usepackage{parskip}
  }{% else
    \setlength{\parindent}{0pt}
    \setlength{\parskip}{6pt plus 2pt minus 1pt}}
}{% if KOMA class
  \KOMAoptions{parskip=half}}
\makeatother
\usepackage{xcolor}
\setlength{\emergencystretch}{3em} % prevent overfull lines
\setcounter{secnumdepth}{-\maxdimen} % remove section numbering
% Make \paragraph and \subparagraph free-standing
\ifx\paragraph\undefined\else
  \let\oldparagraph\paragraph
  \renewcommand{\paragraph}[1]{\oldparagraph{#1}\mbox{}}
\fi
\ifx\subparagraph\undefined\else
  \let\oldsubparagraph\subparagraph
  \renewcommand{\subparagraph}[1]{\oldsubparagraph{#1}\mbox{}}
\fi


\providecommand{\tightlist}{%
  \setlength{\itemsep}{0pt}\setlength{\parskip}{0pt}}\usepackage{longtable,booktabs,array}
\usepackage{calc} % for calculating minipage widths
% Correct order of tables after \paragraph or \subparagraph
\usepackage{etoolbox}
\makeatletter
\patchcmd\longtable{\par}{\if@noskipsec\mbox{}\fi\par}{}{}
\makeatother
% Allow footnotes in longtable head/foot
\IfFileExists{footnotehyper.sty}{\usepackage{footnotehyper}}{\usepackage{footnote}}
\makesavenoteenv{longtable}
\usepackage{graphicx}
\makeatletter
\def\maxwidth{\ifdim\Gin@nat@width>\linewidth\linewidth\else\Gin@nat@width\fi}
\def\maxheight{\ifdim\Gin@nat@height>\textheight\textheight\else\Gin@nat@height\fi}
\makeatother
% Scale images if necessary, so that they will not overflow the page
% margins by default, and it is still possible to overwrite the defaults
% using explicit options in \includegraphics[width, height, ...]{}
\setkeys{Gin}{width=\maxwidth,height=\maxheight,keepaspectratio}
% Set default figure placement to htbp
\makeatletter
\def\fps@figure{htbp}
\makeatother

\KOMAoption{captions}{tableheading}
\makeatletter
\@ifpackageloaded{caption}{}{\usepackage{caption}}
\AtBeginDocument{%
\ifdefined\contentsname
  \renewcommand*\contentsname{Table of contents}
\else
  \newcommand\contentsname{Table of contents}
\fi
\ifdefined\listfigurename
  \renewcommand*\listfigurename{List of Figures}
\else
  \newcommand\listfigurename{List of Figures}
\fi
\ifdefined\listtablename
  \renewcommand*\listtablename{List of Tables}
\else
  \newcommand\listtablename{List of Tables}
\fi
\ifdefined\figurename
  \renewcommand*\figurename{Figure}
\else
  \newcommand\figurename{Figure}
\fi
\ifdefined\tablename
  \renewcommand*\tablename{Table}
\else
  \newcommand\tablename{Table}
\fi
}
\@ifpackageloaded{float}{}{\usepackage{float}}
\floatstyle{ruled}
\@ifundefined{c@chapter}{\newfloat{codelisting}{h}{lop}}{\newfloat{codelisting}{h}{lop}[chapter]}
\floatname{codelisting}{Listing}
\newcommand*\listoflistings{\listof{codelisting}{List of Listings}}
\makeatother
\makeatletter
\makeatother
\makeatletter
\@ifpackageloaded{caption}{}{\usepackage{caption}}
\@ifpackageloaded{subcaption}{}{\usepackage{subcaption}}
\makeatother
\ifLuaTeX
  \usepackage{selnolig}  % disable illegal ligatures
\fi
\usepackage{bookmark}

\IfFileExists{xurl.sty}{\usepackage{xurl}}{} % add URL line breaks if available
\urlstyle{same} % disable monospaced font for URLs
\hypersetup{
  pdftitle={PySchool - Introducing highschool students to the World of Python},
  colorlinks=true,
  linkcolor={blue},
  filecolor={Maroon},
  citecolor={Blue},
  urlcolor={Blue},
  pdfcreator={LaTeX via pandoc}}

\title{PySchool - Introducing highschool students to the World of
Python}
\author{}
\date{}

\begin{document}
\maketitle

\textbf{TLDR;} We used challenges to motivate highschoolers to learn
Python, using a no-installation-required website powered with quarto +
pyodide. They loved it!

\textbf{Poster created by}: Sebastián Flores

\begin{itemize}
\tightlist
\item
  \textbf{Content}: Francisco Alfaro, Sebastián Flores, Tony Rodríguez
\item
  \textbf{Coordination}: Darwin Morales, Patricia Tapia \& Marcelo
  Figueroa.
\item
  \textbf{Volunteers}: Bastián Blandskron, Cristina Verdugo, Cristian
  López, Liliana Garmendia, Valeska Canales, Yialenne Alviña \& María
  Fernanda Villalobos.
\end{itemize}

\subsection{Help us!}\label{help-us}

\begin{itemize}
\tightlist
\item
  \textbf{Share}: Let others know about PySchool!
\item
  \textbf{Give us Feedback}: Let us know what might improve PySchool!
\item
  \textbf{Try it!}: Quarto + Pyodide can help you create interactive
  websites. Attendees won't need to install Python on workshops, talks
  or documentation!
\item
  \textbf{Network and Learn}: We're excited to meet other Python
  enthusiasts and educators to talk about teaching Python, learning
  methods, and new technologies.
\end{itemize}

\subsection{The past}\label{the-past}

\begin{itemize}
\tightlist
\item
  Motivation: events are organized by Python Chile (PyDay and PyCon)
  were aimed towards grownups. We \textbf{dreamed of} an event
  targetting highschool students.
\item
  Philosophy: We know we can't teach Python in a day, but we can show
  how simple and fun coding in Python can be, and motivate them to learn
  Python.
\item
  How: quarto+pyodide allowed us to code a simple website in Spanish
  (pyschool.cl) that does not requires users to install Python! It's
  simpler to use than jupyter notebooks.
\item
  Activities: Intro to Python (1h) and Intro to Data Analysis (1h).
\item
  Events: \# X students, on 3 events done in Valparaíso, Chile.
\item
  Feedback: We had 9.3/10 stars on anonymous surveys.
\end{itemize}

\section{Photos}\label{photos}

\begin{enumerate}
\def\labelenumi{\arabic{enumi}.}
\tightlist
\item
  Workshop at DUOC 1 (didn't work as setup)
\item
  Workshop at DUOC 2 (worked as setup)
\item
  Screenshot of old site
\item
  Screenshot of new site
\end{enumerate}

\subsection{The present}\label{the-present}

\begin{itemize}
\tightlist
\item
  Learn from the feedback:

  \begin{itemize}
  \tightlist
  \item
    Lab setup difficults interaction!
  \item
    Traditional classes are boring, challenges and discussion works
    wonders
  \end{itemize}
\item
  We redid the site as a ``escape room'' where Python puzzles need to be
  solved.
\item
  We will be testing this new approach on the second half of 2025.
\end{itemize}

\subsection{The future?}\label{the-future}

TO DO LIST: - Make it scalable\\
For this project to grow, we need a ``train the trainers'' approach:
make it easily replicable by anyone interested, without needing us to
provide support. - Translate to other languages\\
English? French? Italian?\\
Licence is MIT - feel free what it's valuable to you and translate it,
replicate, mix, whatever. Volunteer if you want to help!

Want the poster pdf?\\
\includegraphics[width=0.5\textwidth,height=\textheight]{./images/qr_code.png}

Want the poster code?\\
\includegraphics[width=0.5\textwidth,height=\textheight]{./images/qr_code.png}

Want to stay in touch?\\
\includegraphics[width=0.5\textwidth,height=\textheight]{./images/qr_code.png}

Want to try it?\\
\includegraphics[width=0.5\textwidth,height=\textheight]{./images/qr_code.png}

.

\begin{figure}[H]

{\centering \includegraphics[width=0.5\textwidth,height=\textheight]{./images/pythonchile.png}

}

\caption{Python Chile}

\end{figure}%

\begin{figure}[H]

{\centering \includegraphics[width=0.5\textwidth,height=\textheight]{./images/duoc_valparaiso.png}

}

\caption{DUOC Valparaíso}

\end{figure}%

\begin{figure}[H]

{\centering \includegraphics[width=0.5\textwidth,height=\textheight]{./images/PSF.png}

}

\caption{PSF}

\end{figure}%

\begin{figure}[H]

{\centering \includegraphics[width=0.5\textwidth,height=\textheight]{./images/cafe_robots_galletas.png}

}

\caption{Robot Cafe Galletas}

\end{figure}%



\end{document}
